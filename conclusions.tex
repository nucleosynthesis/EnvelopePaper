\section{Conclusions} % 1 page
\label{sec:conclusions}

A method of treating the uncertainty due to the choice of background function
as a discrete nuisance parameter has been shown to give good coverage and
small (potentially minimal) bias.

There is an uncertainty about how to penalise functions with higher numbers
of parameters. The results indicate that a penalty based on the p-value
is less biased and gives slightly better coverage than one based on the
Akaike criterion. The approximate p-value correction is easier to implement and
can be applied more widely and so was chosen as the method to use.
However the actual results for the case presented
seem to be relatively independent of the exact value of the penalty.
In general, the choice of the size of the correction must be determined
on a case-by-case basis.

%Also, the bias is related to the correction term and this may mean the
%correction is a function of the application and/or statistics available 
%(?CHECK!).

%The method leads naturally to a recipe for generating toy datasets, given
%a fit to data. These allow for a mixture of the functions used for the
%generation. Hence, this gives a function-independent set of toy
%datasets which encompass the range of possible generating functions for
%a given situation.
%Even if the discrete profiling
%method is not used for actually extracting results
%in an analysis, it might be worth using the toy generation method to measure
%the bias of whichever function is chosen for the analysis fit.
