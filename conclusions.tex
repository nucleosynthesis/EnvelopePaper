\section{Conclusions (1 page)}
\label{sec:conclusions}

A method of treating the uncertainty due to the choice of background function
as a discrete nuisance parameter has been shown to give good coverage and
small (potentially minimal) bias.

There is an uncertainty about how to penalise functions with higher numbers
of parameters. However the actual result for the case presented
seem to be relatively independent of the exact value of the penalty.
Also, the bias is related to the correction term and this may mean the
correction is a function of the application and/or statistics available (?CHECK!).

The method leads naturally to a recipe for generating toy datasets, given
a fit to data. These allow for a mixture of the functions used for the
generation. Hence, this gives a function-independent set of toy
datasets which encompass the range of possible generating functions for
a given situation.
Even if the envelope method is not used for actually extracting results
in an analysis, it might be worth using the toy generation method to measure
the bias of whichever function is chosen for the analysis fit.
