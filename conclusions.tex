\section{Discussion and conclusions} % 1 page
\label{sec:conclusions}
The actual CMS
Higgs to two photon analysis~\cite{ref:introduction:legacy}
is significantly more
complex than the simpler method discussed above. In particular, the 2011
and 2012 data
samples are split into eleven and fourteen categories, respectively, which have
differing signal to background ratios.
Because the categories (by definition) have different selection criteria,
they can have different background shapes.
There is no {\it a priori} reason to make any assumptions that the functions
used in each category should be the same. Hence, each category should be
tested with all functions, in a similar way to the above.

A major complication then arises because there are common systematic effects
across the categories, arising from nuisance parameters in the signal
model.
In the absence of these common nuisance parameters,
the different categories could be profiled independently, each using the
minimum envelope technique to produce an envelope curve per category.
These could then
be summed to give the overall profile curve. However, with common
nuisance parameters, all categories must be profiled at the same time.
Since minimisation code to handle the
discrete nuisance parameter identifying the
function seems difficult, in practical terms, this means that all possible
combinations of each function in each category must be fitted.
The minimum envelope made from the results of all these combinations would
then be found. While this is conceptually straightforward, the actual
naive implementation is prohibitive and approximations were taken in practice,
while retaining the core of the method.

In conclusion,
a method of treating the uncertainty due to the choice of background function
as a discrete nuisance parameter has been shown to give good coverage and
small (potentially minimal) bias.
Although described in terms of a particular application, the method can be
applied very widely as the general type of problem for which this method
is relevant is very common.

There is an uncertainty about how to penalise functions with higher numbers
of parameters. The results indicate that a penalty based on the p-value
is less biased and gives slightly better coverage than one based on the
Akaike criterion. In general, the choice of the size of the correction must be determined
on a case-by-case basis. The approximate p-value correction is easier to implement and
can be applied more widely and so was chosen as the method to use for the CMS Higgs to
two photon analysis.
In the studies presented, the actual results are relatively independent of the exact value of the penalty
applied.


%Also, the bias is related to the correction term and this may mean the
%correction is a function of the application and/or statistics available
%(?CHECK!).

%The method leads naturally to a recipe for generating toy datasets, given
%a fit to data. These allow for a mixture of the functions used for the
%generation. Hence, this gives a function-independent set of toy
%datasets which encompass the range of possible generating functions for
%a given situation.
%Even if the discrete profiling
%method is not used for actually extracting results
%in an analysis, it might be worth using the toy generation method to measure
%the bias of whichever function is chosen for the analysis fit.
