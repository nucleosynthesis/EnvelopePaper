\section{Introduction (1 page)}
\label{sec:introduction}

A very common problem in data analysis is to determine the parameters of
a narrow signal which occurs on a wide, smoothly varying background. If the
background function shape is not know {\it a priori}, then there will be 
some uncertainty in the signal parameters resulting from the uncetainty in
the function to use. This issue is exacerbated in the case of a small signal to
background ratio.

A common approach is to try various different plausible functions and 
determine the spread of the signal parameters when using these functions.
However, these methods tend to have some degree of arbitrariness and so
a new approach is discussed in this paper.
This method was developed as part of the analysis of data at the CMS experiment
following the discovery of the Higgs
boson~\cite{ref:introduction:atlasdis,ref:introduction:cmsdis}.
It was applied to the case where the Higgs decays to two photons, which
results in a narrow signal on a large 
background~\cite{ref:introduction:legacy}.

This method tries to be less 
arbitrary and treats the uncertainty associated with the
background shape function in a way
which is more like the other
uncertainties associated with the measurement, i.e. as a systematic error
which is handled as a nuisance parameter.
There are two major new parts to this approach, namely the method for
treating the choice of function as a nuisance parameter, and how to compare 
functions with different numbers of parameters.


The concept of this approach is described in Section~\ref{sec:concept}.
The application of the method to functions with the same number of parameters
is described in Section~\ref{sec:functions} and to functions with different
numbers of parameters in Section~\ref{sec:correction}. Further discussion on
the method, including its practical application to the real-world problem of
the Higgs measurements, is given in Section~\ref{sec:discussion}.

Within this paper, twice the negative of the logarithm of the likelihood
function is denoted by 2NLL. The fits discussed are binned and the 
likelihood used for each bin is the likelihood ratio to the best
possible likelihood given the observed data, i.e.
\begin{equation}
{\rm 2NLL} = \mu_i - n_i + n_i \log\left(\frac{n_i}{\mu_i}\right)
\label{eqn:introduction:def2NLL}
\end{equation}

FREQUENTIST UNLESS OTHERWISE STATED (ALSO SEE DISCUSSION)?
