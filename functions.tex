\section{Using functions with equal numbers of parameters (5 pages)}
\label{sec:functions}

The simplest application of the envelope method is to the case where all
functions used have the same number of parameters. This avoids several extra
complications which are discussed in the next section.

\subsection{Function definitions}
\label{sec:functions:function}

ONLY TWO PARAMETER FUNCTIONS; POL1 ETC

\subsection{Example case}
\label{sec:functions:example}
DEFINITION OF ``ORIGINAL'' DATASET

PLOTS OF BEST FITS OF FUNCTIONS TO DATASET

ENVELOPE FOR ALL TWO-PARA FUNCTIONS


\subsection{Bias and coverage}
\label{sec:functions:coverage}
SIMPLE ``BEST FIT'' TOY.
Pseudo-experiment (``toy'')

BIAS AND COVERAGE WITH AND WITHOUT ENVELOPE METHOD, I.E. SHOW SYSTEMATIC
ERROR CONTRIBUTION IS NECESSARY

ASIMOV RESULTS AS PART OF DISCUSSION ON BIAS?

UNBINNED MENTIONED ONLY IN PASSING


\subsection{Toy generation}
\label{sec:functions:toys}

WHAT IS A BETTER TERM THAN ``TOYS''?

1. USING SINGLE FUNCTION (GIVING BIAS)

2. USING BAYESIAN PROBABILITY MIX OF FUNCTIONS (NO BIAS)

3. USING FREQUENTIST FRACTIONAL CONTRIBUTIONS OF FUNCTIONS (BIAS???)
