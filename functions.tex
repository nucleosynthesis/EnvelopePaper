\section{Using functions with equal numbers of parameters (5 pages)}
\label{sec:functions}

The simplest application of the envelope method is to the case where all
functions used have the same number of parameters. This avoids several extra
complications which are discussed in the next section.

\subsection{Function definitions}
\label{sec:functions:function}

The first study presented uses four functions, each of which has two parameters.
These functions are chosen as they, and their higher order equivalents,
are feasible representations of the background shape seen in the Higgs
analysis. The functions are detailed below; in each case $p_0$ and $p_1$ are
the two parameters.
\begin{enumerate}
\item
``Power law''; $f(x) = p_0 x^{p_1}$.
\item
``Exponential''; $f(x) = p_0 e^{p_1x}$.
\item
``Laurent''; $f(x) = p_0/x^4 + p_1/x^5$.
\item
``Polynomial''; $f(x) = p_0 + p_1 x$.
\end{enumerate}


\subsection{Example case}
\label{sec:functions:example}

In the Higgs analysis, the discrete profiling method is obviously applied to
the actual data taken by the CMS experiment; specifically the likelihood fit
is done to the invariant mass spectrum of the pairs of two photons.
However, for the purposes of this
paper, the ``data'' used to illustrate the method are generated by a Monte Carlo
method from a smooth background
function which is similar in shape and magnitude to the
real Higgs data sample used. However, it is emphasised that this is not the 
actual data sample and so 
none of the detailed results presented below can be used to deduce any
properties of the Higgs boson iself.
This original dataset was generated in 320 bins. It included XXX signal events
generated according to a Gaussian distribution with a mean of 125\,GeV and a 
width of YYY\,GeV. In the following, the signal strength results are given in
terms of the relative strength $\mu$, 
meaning the measured size of the signal relative
to the generated value.

The four two-parameter functions mentioned above were each 
separately fitted to this
original dataset. A Gaussian signal component was also included, where the
mean and width of the Gaussian were fixed to the same values as used to
generate the events.
The magnitude of the signal Gaussian and both parameters in the
background function were determined from the fit.
The results are shown in figure~\ref{fig:functions:bestfits}.
It is clear the first order polynomial does not fit well at all, while the 
other three functions appear to give reasonable fits.
%
\begin{figure}[tbp]
\centering
\includegraphics[width=0.45\textwidth]{functions/BestFits.pdf}
\caption{Best fits of the four two-parameter functions (described in the
text). The Laurent function is effectively identical to the power law function
and so is hidden below the power law line.}
\label{fig:functions:bestfits}
\end{figure}

The profile scan as a function of the relative signal strength $\mu$
between $-1$ and 2
for the four functions is shown in
figure~\ref{fig:functions:profiles}.
The absolute minimum occurs for the power law function at a relative signal 
strength of around $\mu \approx 1.1$. If just considering this function,
then the measured value of the relative strength would be quoted as
$\mu = 1.1 \pm 0.5$, determining the error from $\Delta({\rm 2NLL}) = 1$.
The exponential function gives a very similar 2NLL value
at its best fit, but gives a relative signal strength of $\mu = 0.5 \pm 0.4$.
CHECK THESE VALUES!
The Laurent result of $\mu = 1.1 \pm 0.5$
is very similar to the power law in terms of the
relative signal strength, but with a slightly
worse 2NLL value. The polynomial 2NLL value is too large to appear in the plot
but the best fit $\mu$ value is at the lower edge of the profile scan range
and so does not have a true minimum.
The fact that the different functions can give different best fit values
is a direct example of the systematic error associated
with the choice of function.
%
\begin{figure}[tbp]
\centering
\includegraphics[width=0.45\textwidth]{functions/Profiles.pdf}
\caption{Profile 2NLL scans for the four functions discussed in the text.
The polynominal function is always above the top of the 2NLL scale shown in this
figure. SHOULD THIS HAVE AN ENVELOPE CURVE, OR A SEPARATE PLOT WITH JUST
THE ENVELOPE PLUS 1SIGMA CONSTRUCTION LINES? IN NEXT PLOT}
\label{fig:functions:profiles}
\end{figure}

The envelope around these functions is shown in
figure~\ref{fig:functions:envelope}.
The best fit is clearly still $\mu \approx 1.1$ from the power law
but now the error is enlarged by the exponential contribution to the
envelope on the lower side of the scan. Hence, taking all four functions into
account, the value which would now be quoted would be 
$\mu = 1.1^{+0.5}_{-1.0}$. The enlarged error is a direct reflection of the
systematic error arising from the function uncertainty. Note,
it is clear that the very bad fit of the polynomial
means it plays no role in the envelope and so this function is 
``automatically'' ignored by the method,
without requiring any arbitrary criterion for
including it or not.
%
\begin{figure}[tbp]
\centering
\includegraphics[width=0.45\textwidth]{functions/Profiles.pdf}
\caption{Profile 2NLL envelope for the four two-parameter function fits.
The lines shown the best fit and the errors determined from
$\Delta({\rm 2NLL})=1$.
PLACEHOLDER}
\label{fig:functions:envelope}
\end{figure}


\subsection{Bias and coverage}
\label{sec:functions:coverage}

The discrete profile method was tested for bias and coverage 
using a large ensemble of
pseudo-experiments (``toys''). For each of these, a background and signal 
dataset were generated using a Monte Carlo technique. The background function
was chosen to be one of the four two-parameter functions discussed above, with
the parameters set to their best fit values. In addition, a further set of
toy datasets were generated based on the same
smooth function and Gaussian as for the original dataset.

WHAT SIGNAL STRENGTH?

In all cases, the resulting toy datasets were then treated identically to the
original dataset, with a fit for the relative signal strength using
each of the four functions as background, and then forming the envelope.
For each ensemble of datasets, the mean pull of the relative
signal strength was determined. Figure~\ref{fig:functions:firstorderbias}
shows the result of these fits. It is seen that the envelope gives a very small
bias for all cases, which is not the case for any of the specific functions.
%
\begin{figure}[tbp]
\centering
\includegraphics[width=0.45\textwidth]{functions/FirstOrderFunctions.pdf}
\caption{Average pull when fitting with each function as background and when
using the envelope. The second, third and fourth plots shows the results
when the generating background function is power law, exponential and Laurent,
respectively. The top plot show the result when the best-fit function at each
value of $\mu$ is used to generate toys; this means the exponential function
below $\mu = 0.8$ and the power law function above this value. In all cases,
the polynomial gives values outside the range of these plots.}
\label{fig:functions:firstorderbias}
\end{figure}

The coverage was also determined from the same fits. The difference of the
2NLL between the best fit value and the true value of the relative signal 
strength was found and the fraction of times this was less than 1 or 4 was
determined. Figure~\ref{fig:functions:firstordercoverage}
shows the result of these studies. It is seen that the envelope gives good
coverage for all cases, which is again not the case for any of the specific 
functions.
%
\begin{figure}[tbp]
\centering
\includegraphics[width=0.45\textwidth]{functions/FirstOrderFunctions_Coverage.pdf}
\caption{Fraction of relative signal strength within the ``one sigma'' and
``two sigma'' range when fitting with each function as background and when
using the envelope. The dashed horizontal
lines represent the 0.683 and 0.954 fractions
expected for one and two sigma, respectively.
The second, third and fourth plots shows the results
when the generating background function is power law, exponential and Laurent,
respectively. The top plot show the result when the best-fit function at each
value of $\mu$ is used to generate toys.}
\label{fig:functions:firstordercoverage}
\end{figure}


ASIMOV RESULTS AS PART OF DISCUSSION ON BIAS?

UNBINNED MENTIONED ONLY IN PASSING


\subsection{Toy generation}
\label{sec:functions:toys}

WHAT IS A BETTER TERM THAN ``TOYS''?

1. USING SINGLE FUNCTION (GIVING BIAS)

2. USING BAYESIAN PROBABILITY MIX OF FUNCTIONS (NO BIAS)

3. USING FREQUENTIST FRACTIONAL CONTRIBUTIONS OF FUNCTIONS (BIAS???)
