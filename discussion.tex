%\section{Discussion (3 pages)}
%\label{sec:discussion}

%\subsection{Application to real world case}
\section{Application to real world case}
\label{sec:discussion}
\label{sec:discussion:higgs}

HOW APPLIED TO HIGGS ANALYSIS.

The actual $H \rightarrow \gamma\gamma$ analysis is significantly more
complex than the simplified version used here. In particular, the 2011 
and 2012 data
samples are split into five and nine categories, respectively, which have
differing signal:background ratios.
Because the categories (by definition) have different selection criteria,
they can have different background shapes.
There is no {\it a priori} reason to make any assumptions that the functions
used in each category should be the same. Hence, each category should be
tested with all functions, in a similar way to the above.

A major complication then arises because there are common systematic effects
across the categories, arising from nuisance parameters in the signal
model.
In the absence of these common nuisance parameters,
the different categories could be profiled independently, using the
minimum envelope technique to produce a curve per category. These could then
be summed to give the overall profile curve. However, with common
nuisance parameters, all categories must be profiled at the same time.
Since minimisation code to handle the 
discrete nuisance parameter identifying the
function seems difficult, in practical terms, this means that all possible
combinations of each function in each category must be fitted.
The minimum envelope made from the results of all these combinations would
then be found. While this is conceptually straightforward, the actual
implementation is probably prohibitive. For example, using the 16 functions
discussed in this paper, there would be $16^5 \sim$ 1M combinations of functions
to be fitted for the 2011 data sample and $16^9 \sim$ 70B combinations for the
2012 data sample.

DETAILS ON HOW HANDLED FROM NICK

%\subsection{Envelope smoothing}
%\label{sec:discussion:smoothing}

%DISCRETE FUNCTIONS MEAN NON-SMOOTH ENVELOPE

%APPLY SMOOTHING USING [SUM 2NLL EXP(-2NLL/2)]/[SUM EXP(-2NLL/2)]?


%\subsection{Bayesian equivalent method}
%\label{sec:discussion:bayesian}
