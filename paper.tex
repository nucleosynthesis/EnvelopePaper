\pdfoutput=1 % only if pdf/png/jpg images are used
\documentclass{JINST}
\usepackage{xspace}
\let\ifpdf\relax
\title{Handling uncertainties in background shapes: the discrete profiling method}

\author{P.~D.~Dauncey$^a$\thanks{Corresponding author.},
M.~Kenzie$^b$, N.~Wardle$^b$ and G.~J.~Davies$^a$\\
\llap{$^a$}Department of Physics, Imperial College London, Prince Consort Road, London, SW7 2AZ, UK.\\
\llap{$^b$}CERN, CH-1211 Geneva 23, Switzerland.\\
E-mail: \email{P.Dauncey@imperial.ac.uk}}


\abstract{
A common problem in data analysis is that the precise parametrisation, as well as the parameter values,
of the underlying model which should describe a dataset is not always {\it a priori} known. In these cases some
extra uncertainty must be assigned to the extracted parameters of interest because of the lack of knowledge of the functional form of the model.
A method for assigning an error from this cause is presented. It is based on
treating the lack of knowledge of the functional form as a discrete nuisance parameter which is
profiled in an equivalent way to continuous nuisance parameters. This
effectively means an ``envelope'' is found which encompasses the lowest
log-likelihood values for any given
parameter of the model. The bias and coverage of this method are shown to be good when applied to
a realistic example.
}

\keywords{Analysis and statistical methods; Simulation methods and programs}

\begin{document}
\newcommand{\nll}{\ensuremath{\Lambda}\xspace}

\input introduction.tex
\input concept.tex
\input functions.tex
\input correction.tex
\input discussion.tex
\input conclusions.tex

\acknowledgments
We thank Chris Seez and Louis Lyons for informative discussions.
This work was partially supported by the Science and Technology Facilities
Council, UK.

\bibliographystyle{JHEP}
\bibliography{paper}
%\documentclass[11pt,a4paper]{article}
\usepackage{graphicx}
\usepackage{rotating}

\oddsidemargin=0pt           % No extra space wasted after first inch.
\evensidemargin=0pt          % Ditto (for two-sided output).
\topmargin=0pt               % Same for top of page.
\headheight=0pt              % Don't waste any space on unused headers.
\headsep=0pt
\textwidth=16cm              % Effectively controls the right margin.
\textheight=24cm

\begin{document}
\title{
\Large \bf Handling uncertainties in background shapes: the envelope method}
\author{P.~D.~Dauncey$^1$, G.~J.~Davies$^1$, M.~Kenzie$^2$, N.~Wardle$^2$\\
$^1$Imperial College London, $^2$CERN}

\maketitle

\tableofcontents
\newpage

\input introduction.tex
\input concept.tex
\input functions.tex
\input correction.tex
\input discussion.tex
\input conclusions.tex
\input references.tex

\end{document}


\end{document}
