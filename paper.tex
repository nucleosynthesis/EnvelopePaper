\pdfoutput=1 % only if pdf/png/jpg images are used
\documentclass{JINST}

\title{Handling uncertainties in background shapes: the envelope method}

\author{P.~D.~Dauncey$^a$\thanks{Corresponding author.},
M.~Kenzie$^b$, N.~Wardle$^b$ and G.~J.~Davies$^a$\\
\llap{$^a$}Department of Physics, Imperial College London, London, UK.\\
\llap{$^b$}CERN, Geneva, Switzerland.\\
E-mail: \email{P.Dauncey@imperial.ac.uk}}


\abstract{
A common problem in data analysis is measuring a narrow signal on a smoothly
varying background. In cases where the background functional form is not
{\it a priori} known, then some extra uncertainty must be assigned to the
signal parameters because of this lack of knowledge.
A method for assigning an error from this cause is presented. It is based on
treating the function uncertainty as a discrete nuisance parameter and finding
an ``envelope'' which encompasses the lowest log-likelihood values for any given
signal parameter. The bias and coverage of this method are shown to be good.
}

\keywords{Analysis and statistical methods; Simulation methods and programs}

\begin{document}

\input introduction.tex
\input concept.tex
\input functions.tex
\input correction.tex
\input discussion.tex
\input conclusions.tex

\acknowledgments
We thank Chris Seez and Louis Lyons for informative discussions.
This work was partially supported by the Science and Technology Facilities
Council, UK.

\bibliographystyle{JHEP}
\bibliography{paper}
%\pdfoutput=1 % only if pdf/png/jpg images are used
\documentclass{JINST}
\usepackage{xspace}
\usepackage{subfigure}
\let\ifpdf\relax
\title{Handling uncertainties in background shapes: the discrete profiling method}

\author{P.~D.~Dauncey$^a$\thanks{Corresponding author.},
M.~Kenzie$^b$, N.~Wardle$^b$ and G.~J.~Davies$^a$\\
\llap{$^a$}Department of Physics, Imperial College London, Prince Consort Road, London, SW7 2AZ, UK.\\
\llap{$^b$}CERN, CH-1211 Geneva 23, Switzerland.\\
E-mail: \email{P.Dauncey@imperial.ac.uk}}


\abstract{
A common problem in data analysis is that the functional form, as well as the parameter values,
of the underlying model which should describe a dataset is not known {\it a priori}. In these cases some
extra uncertainty must be assigned to the extracted parameters of interest due to lack of exact knowledge of the functional form of the model.
A method for assigning an appropriate error is presented. The method is based on
considering the choice of functional form as a discrete nuisance parameter which is
profiled in an analogous way to continuous nuisance parameters. The bias and coverage of this method are shown to be good when applied to
a realistic example.
}

\keywords{Analysis and statistical methods; Simulation methods and programs}

\begin{document}
\newcommand{\nll}{\ensuremath{\Lambda}\xspace}

\input introduction.tex
\input concept.tex
\input functions.tex
\input correction.tex
\input discussion.tex
\input conclusions.tex

\acknowledgments
We thank Chris Seez and Louis Lyons for informative discussions.
This work was partially supported by the Science and Technology Facilities
Council, UK.

\bibliographystyle{JHEP}
\bibliography{paper}
%\pdfoutput=1 % only if pdf/png/jpg images are used
\documentclass{JINST}
\usepackage{xspace}
\usepackage{subfigure}
\let\ifpdf\relax
\title{Handling uncertainties in background shapes: the discrete profiling method}

\author{P.~D.~Dauncey$^a$\thanks{Corresponding author.},
M.~Kenzie$^b$, N.~Wardle$^b$ and G.~J.~Davies$^a$\\
\llap{$^a$}Department of Physics, Imperial College London, Prince Consort Road, London, SW7 2AZ, UK.\\
\llap{$^b$}CERN, CH-1211 Geneva 23, Switzerland.\\
E-mail: \email{P.Dauncey@imperial.ac.uk}}


\abstract{
A common problem in data analysis is that the functional form, as well as the parameter values,
of the underlying model which should describe a dataset is not known {\it a priori}. In these cases some
extra uncertainty must be assigned to the extracted parameters of interest due to lack of exact knowledge of the functional form of the model.
A method for assigning an appropriate error is presented. The method is based on
considering the choice of functional form as a discrete nuisance parameter which is
profiled in an analogous way to continuous nuisance parameters. The bias and coverage of this method are shown to be good when applied to
a realistic example.
}

\keywords{Analysis and statistical methods; Simulation methods and programs}

\begin{document}
\newcommand{\nll}{\ensuremath{\Lambda}\xspace}

\input introduction.tex
\input concept.tex
\input functions.tex
\input correction.tex
\input discussion.tex
\input conclusions.tex

\acknowledgments
We thank Chris Seez and Louis Lyons for informative discussions.
This work was partially supported by the Science and Technology Facilities
Council, UK.

\bibliographystyle{JHEP}
\bibliography{paper}
%\pdfoutput=1 % only if pdf/png/jpg images are used
\documentclass{JINST}
\usepackage{xspace}
\usepackage{subfigure}
\let\ifpdf\relax
\title{Handling uncertainties in background shapes: the discrete profiling method}

\author{P.~D.~Dauncey$^a$\thanks{Corresponding author.},
M.~Kenzie$^b$, N.~Wardle$^b$ and G.~J.~Davies$^a$\\
\llap{$^a$}Department of Physics, Imperial College London, Prince Consort Road, London, SW7 2AZ, UK.\\
\llap{$^b$}CERN, CH-1211 Geneva 23, Switzerland.\\
E-mail: \email{P.Dauncey@imperial.ac.uk}}


\abstract{
A common problem in data analysis is that the functional form, as well as the parameter values,
of the underlying model which should describe a dataset is not known {\it a priori}. In these cases some
extra uncertainty must be assigned to the extracted parameters of interest due to lack of exact knowledge of the functional form of the model.
A method for assigning an appropriate error is presented. The method is based on
considering the choice of functional form as a discrete nuisance parameter which is
profiled in an analogous way to continuous nuisance parameters. The bias and coverage of this method are shown to be good when applied to
a realistic example.
}

\keywords{Analysis and statistical methods; Simulation methods and programs}

\begin{document}
\newcommand{\nll}{\ensuremath{\Lambda}\xspace}

\input introduction.tex
\input concept.tex
\input functions.tex
\input correction.tex
\input discussion.tex
\input conclusions.tex

\acknowledgments
We thank Chris Seez and Louis Lyons for informative discussions.
This work was partially supported by the Science and Technology Facilities
Council, UK.

\bibliographystyle{JHEP}
\bibliography{paper}
%\input{paper.bib}

\end{document}


\end{document}


\end{document}


\end{document}
